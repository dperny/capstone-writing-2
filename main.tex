\documentclass{sig-alternate-05-2015}

\begin{document}

\title{Capstone Writing 2}
\author{Drew Erny}
\date{November 29, 2015}

\numberofauthors{1}

\author{
  \alignauthor
  Drew Erny
  \affaddr{302 Reed St.}
  \affaddr{Apt 22A}
  \affaddr{Tuscaloosa, AL 35401}
  \email{dperny@crimson.ua.edu}
}

\maketitle
\begin{abstract}
This paper discusses the ethical problem of software engineers developing
software for use in government mass surveillance. Since NSA contractor Edward
Snowden leaked documents revealing the extent of US government surveillance in
2013, software developers have been discussing their professional and ethical
obligation to oppose mass surveillance. This paper discusses the underlying
technology that powers mass surveillance, the general ethical arguments for and
against, and programmer's responsibilities when confronted with mass
surveillance.
\end{abstract}

\section{Introduction}
In June of 2013, NSA contractor Edward Snowden obtained and leaked a large
number of classified documents, in part detailing, among other things,
unconstitutional mass-surveillance programs conducted by the NSA. These
programs collect and mine data on an unprecedented scale in an attempt to
discover and terminate terrorist plots before they happen. However, there was
widespread backlash among civil liberties advocates, who argue that mass
surveillance infringes upon the right to privacy and stifles the liberal
democracy it attempts to uphold. In the wake of the leaks, discussion ignited
among computer science professionals about their ethical responsibilities
toward such programs. 

\section{What is Surveillance}

For a comprehensive understanding of the ethical issues of surveillance, we
first have to understand what mass surveillance is, what it's used for and the
technology that powers it. Privacy International defines "Mass Surveillance" as
"... the subjection of a population or significant component of a group to
indiscriminate monitoring" \cite{website:privint}. In other words, mass
surveillance is monitoring everybody. 

Mass surveillance, long a tool of autocratic regimes like China and the USSR, 
came to prominence in democratic governments the wake of early and mid-2000s 
terror attacks in the United States and the United Kingdom. In the US, it was
the passage of the USA PATRIOT Act in 2001 that provided the legal groundwork
for the NSA's mass surveillance programs.

Technology for mass surveillance takes many forms and names, but perhaps the
most infamous is the NSA program codenamed PRISM. PRISM, one of the first top
secret surveillance programs exposed by whistleblower Edward Snowden and
featured in a \textit{Guardian} article from June 2013 \cite{guardian}, is a
program to collect massive amounts of bulk data from major internet companies.
PRISM works by tapping directly into the servers of internet giants like
Google and Apple. NSA agents can then analyze any data from these companies
without the requirement and length waiting period of a warrant.

Another NSA program, called XKEYSCORE and outlined in July 2015 by The
Intercept, is used to intercept and store user data collected at  approximately
150 field sites around the world \cite{xkeyscore}. Referred to internally at
the NSA as the "widest reaching" system, XKEYSCORE not only collects but allows
agents to search collected data and retrieve anything from phonecalls and Skype
conversations to usernames and passwords.

Once all of this data is collected, specific targets can be picked out an
analyzed by hand, or the data can be mined using computer analysis to find
noteworthy data points. Data mining involves the use of
sophisticated computer algorithms to sort out terrorist threats from the troves
of data collected. \cite{mining}. Theoretically, the use of data mining can
identify and prevent terrorist attacks before they happen.

\section{Ethics of Surveillance}

As outlined in the previous section, mass surveillance is a powerful tool, and
like most powerful tools, its use can come with cost. 

Neil M. Richards mentions in his essay "The Dangers of Surveillance,"
\cite{richards} we often find it hard to pin down exactly what harm
surveillance does us. Richards mentions that we draw much of our distrust from
literary tales like Orwell's 1984, but admits that even the most ardent critics
do not consider NSA programs analogous to Orwell's Big Brother.  Instead, he
presents two angles from which surveillance does harm. 

The first angle is that surveillance constitutes a violation of intellectual
privacy. Richards argues that intellectual privacy, -- that is, feeling that
our thoughts are not being monitored -- is important because the lack of
privacy drives us to avoid behaviors that might be considered deviant or
undesirable.

The second angle is that surveillance, at its core, is about power over those
being watched. Surveillance is always for a purpose, and many purposes are less
than benign. For example, Richards points to the case of Dr. Martin Luther King
Jr., who was blackmailed over marital indiscretions discovered in the course of
FBI surveillance programs. He also points to the case of World War II, when
census records were unsealed and used to intern Japanese- and German-Americans.

Richards is not an academic within the field, but within the industry, many
programmers also argue that this collection is unethical. For example, Rogaway
argues in his Statement of Condemnation \cite{rogaway} that "Mass surveillance
is intimidating, abuse-prone, and anti-democratic." He uses his position as a
cryptographer as a basis for the legitimacy of his condemnation of the US 
surveillance apparatus.

\section{Responsibility of the Programmer}

The Association for Computing Machinery, a professional organization for
computing professionals, addresses the issue of privacy in its Code of Ethics
\cite{ethics}, in section 1.7, "Respect the privacy of others." In it, ACM
instructs members to collect the minimum amount of data, adhere to strict data
collection policies, and treat user data with the "...strictest
confidentiality, except in cases where it is evidence for violation of law...".
It is unclear whether or not the ACM Code of Ethics permits collection of data
for government authorities. Thus, without clear guidance from our forebears in
a professional code of ethics, we are left with the question of how, as
computer professionals, to approach the issue of mass surveillance.

Mathematician Tom Leinster argues in an article in \textit{New Scientist}
\cite{leinster} that mathematicians have a responsibility to not participate in
mass-surveillance by depriving agencies of their talent. Similarly, Alexander
Beilinson writes in a letter \cite{beilinson} to the editors of \textit{Notices
of the AMS} that the AMS should sever ties with the NSA in the wake of the
revelations about mass surveillance. This is one possible answer to the ethical
question. In this case, ethics don't have to come from systemic codes or rules;
instead, an industry-wide ethical position on mass surveillance can develop
from the ground up.

However, Leinster's approach has one substantial drawback: the NSA is the
largest employer of mathematicians in the United States. Refusing to do
business with them substantially hampers one's career and livelihood. The
choice between working in the field you love and working on ethically ambiguous
projects is less than clear cut when it comes time to make the decision.

Edward Snowden, a professional in the software field, shows us an alternative
ethical approach. When confronted with the ethical question of 
likely-unconstitutional mass surveillance programs, Snowden chose to release
classified documents outlining those programs to journalists, starting a
national conversation on the topic. The question of whether or not his
disclosures were ethical and justified is a hotly debated topic both within the
industry and outside, and the answer to that question is closely connected to
the question of what computer science professionals should do when confronted
with mass surveillance.

Snowden's approach is even more consequential than Leinster's. Blowing the
whistle on government programs left him in exile in Russia, and facing charges
that could send him to jail for the rest of his life.

\section{Conclusion}

Mass surveillance is a powerful technology, and the Snowden leaks have led to a
massive national discussion on ethics of such programs, and a smaller,
intra-industry discussion on how to handle the ethics of those programs. As you
can see, mass surveillance can be very fraught with ethical problems, and the
solutions are less than clear.

\bibliographystyle{abbrv}
\bibliography{main}

\end{document}

% vim: tw=79 formatoptions+=t wm=2

